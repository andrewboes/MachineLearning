\documentclass{article}
\usepackage[utf8]{inputenc}
\usepackage{comment}

\usepackage{amsmath}% http://ctan.org/pkg/amsmath

\title{535 HW3}

\begin{document}




\textbf{Q4} Given a binary string s we need to compute y where y=0 if s has an even number 1's and y=1 if s has an odd number of ones. We will manually calculate the weights for this task using an LSTM with the following equations:


\begin{equation}
    i_t = \sigma(w_{ix}x_t + w_{i_h}h_{t-1}+b_i) 
\end{equation}

\begin{equation}
    f_t = \sigma(w_{f_x}x_t + w_{f_h}h_{t-1}+b_f) 
\end{equation}

\begin{equation}
    o_t = \sigma(w_{ox}x_t + w_{oh}h_{t-1}+b_o) 
\end{equation}

\begin{equation}
    g_t = \tanh(w_{g_x}x_t + w_{g_h}h_{t-1}+b_g) 
\end{equation}

\begin{equation}
    c_t = f_t c_{t-1} + i_t g_t
\end{equation}

\begin{equation}
    h_t = o_t\tanh(c_t)
\end{equation}

As we are using an LSTM, the only thing we need to know is the parity of the previous values and the current value. The table below shows the different combinations and their output ($h_t$). 

\begin{center}\begin{tabular}{ |c|c|c|c| } 
\hline
$h_{t-1}$ & $x_t$ & $h_t$ \\
\hline
0 & 0 & 0 \\ 
0 & 1 & 1 \\ 
1 & 0 & 1 \\ 
1 & 1 & 0 \\ 
\hline
\end{tabular}\end{center}

Let's make $f_t$ always zero as we don't need the cell state, only $h_{t-1}$. This makes w_fx, w_fh=0 and b_f=-10


\begin{center}\begin{tabular}{ |c|c|c|c|c|c| } 
\hline
$h_{t-1}$ & $x_t$ & $h_t$ & $i_t$ & $i_t$ \\
\hline
0 & 0 & 0 & $\sigma(w_{i_x}*0 + w_{i_h}*0+b_i)$ &$\sigma(b_i)$ \\ 
0 & 1 & 1 & $\sigma(w_{i_x}*1 + w_{i_h}*0+b_i)$ & $\sigma(w_{i_x}+b_i)$ \\ 
1 & 0 & 1 & $\sigma(w_{i_x}*0 + w_{i_h}*1+b_i)$ & $\sigma(w_{i_h}+b_i)$ \\ 
1 & 1 & 0 & $\sigma(w_{i_x}*1 + w_{i_h}*1+b_i)$& $\sigma(w_{i_x} + w_{i_h}+b_i)$ \\ 
\hline
\end{tabular}\end{center}

\begin{center}\begin{tabular}{ |c|c|c|c|c|c| } 
\hline
$h_{t-1}$ & $x_t$ & $h_t$ & $o_t$ & $o_t$ \\
\hline
0 & 0 & 0 & $\sigma(w_{o_x}*0 + w_{o_h}*0+b_o)$ &$\sigma(b_o)$ \\ 
0 & 1 & 1 & $\sigma(w_{o_x}*1 + w_{o_h}*0+b_o)$ & $\sigma(w_{o_x}+b_o)$ \\ 
1 & 0 & 1 & $\sigma(w_{o_x}*0 + w_{o_h}*1+b_o)$ & $\sigma(w_{o_h}+b_o)$ \\ 
1 & 1 & 0 & $\sigma(w_{o_x}*1 + w_{o_h}*1+b_o)$& $\sigma(w_{o_x} + w_{o_h}+b_o)$ \\ 
\hline
\end{tabular}\end{center}


\begin{center}\begin{tabular}{ |c|c|c|c|c|c| } 
\hline
$h_{t-1}$ & $x_t$ & $h_t$ & $g_t$ & $g_t$ \\
\hline
0 & 0 & 0 & $\tanh(w_{g_x}*0 + w_{g_h}*0+b_g) $ &$\tanh(b_g)$ \\ 
0 & 1 & 1 &  $\tanh(w_{g_x}*1 + w_{g_h}*0+b_g) $ & $\tanh(w_{g_x}+b_g)$ \\ 
1 & 0 & 1 & $\tanh(w_{g_x}*0 + w_{g_h}*1+b_g)$ & $\tanh(w_{g_h}+b_g)$ \\ 
1 & 1 & 0 & $\tanh(w_{g_x}*1 + w_{g_h}*1+b_o)$& $\tanh(w_{g_x} + w_{g_h}+b_g)$ \\ 
\hline
\end{tabular}\end{center}

\begin{center}\begin{tabular}{ |c|c|c|c|c|c| } 
\hline
$h_{t-1}$ & $x_t$ & $h_t$ & $c_t= 0* c_{t-1} + i_t g_t$ \\
\hline
0 & 0 & 0 & $\sigma(b_i)\tanh(b_g)$ \\ 
0 & 1 & 1 & $\sigma(w_{i_x}+b_i)\tanh(w_{g_x}+b_g)$ \\ 
1 & 0 & 1 & $\sigma(w_{i_h}+b_i)\tanh(w_{g_h}+b_g)$\\ 
1 & 1 & 0 & $\sigma(w_{i_x}+w_{i_h}+b_i)\tanh(w_{g_x} + w_{g_h}+b_g)$ \\ 
\hline
\end{tabular}\end{center}

\begin{center}\begin{tabular}{ |c|c|c|c|c|c| } 
\hline
$h_{t-1}$ & $x_t$ & $h_t$ & $h_t = o_t\tanh(c_t)$ \\
\hline
0 & 0 & 0 & $0=\sigma(b_o)\tanh(\sigma(b_i)\tanh(b_g))$ \\ 
0 & 1 & 1 & $1=\sigma(w_{o_x}+b_o)\tanh(\sigma(w_{i_x}+b_i)\tanh(w_{g_x}+b_g))$ \\ 
1 & 0 & 1 & $1=\sigma(w_{o_h}+b_o)\tanh(\sigma(w_{i_h}+b_i)\tanh(w_{g_h}+b_g))$\\ 
1 & 1 & 0 & { $0=\sigma(w_{o_x}+w_{o_h}+b_o)\tanh(\sigma(w_{i_x}+w_{i_h}+b_i)\tanh(w_{g_x} + w_{g_h}+b_g))$} \\ 
\hline
\end{tabular}\end{center}

We need $\sigma(b_o)\tanh(\sigma(b_i)\tanh(b_g))=0$ so we'll set $b_0=-10$ 

\begin{center}\begin{tabular}{ |c|c|c|c|c|c| } 
\hline
$h_{t-1}$ & $x_t$ & $h_t$ & $h_t = o_t\tanh(c_t)$ \\
\hline
0 & 0 & 0 & $0=\sigma(-10)\tanh(\sigma(b_i)\tanh(b_g))$ \\ 
0 & 1 & 1 & $1=\sigma(w_{o_x}-10)\tanh(\sigma(w_{i_x}+b_i)\tanh(w_{g_x}+b_g))$ \\ 
1 & 0 & 1 & $1=\sigma(w_{o_h}-10)\tanh(\sigma(w_{i_h}+b_i)\tanh(w_{g_h}+b_g))$\\ 
1 & 1 & 0 & { $0=\sigma(w_{o_x}+w_{o_h}-10)\tanh(\sigma(w_{i_x}+w_{i_h}+b_i)\tanh(w_{g_x} + w_{g_h}+b_g))$} \\ 
\hline
\end{tabular}\end{center}


\begin{center}\begin{tabular}{ |c|c|c|c|c|c| } 
\hline
$h_{t-1}$ & $x_t$ & $h_t$ & $h_t = o_t\tanh(c_t)$ \\
\hline
0 & 0 & 0 & $0=\sigma(-10)\tanh(\sigma(0)\tanh(0))$ \\ 
0 & 1 & 1 & $1=\sigma(20-10)\tanh(\sigma(10+0)\tanh(10+0))$ \\ 
1 & 0 & 1 & $1=\sigma(20-10)\tanh(\sigma(10+0)\tanh(10+0))$\\ 
1 & 1 & 0 & { $0=\sigma(20+20-10)\tanh(\sigma(10+10+0)\tanh(10 +10 +0))$} \\ 
\hline
\end{tabular}\end{center}

\begin{comment}
\begin{center}\begin{tabular}{ |c|c|c|c|c|c| } 
\hline
$h_{t-1}$ & $x_t$ & $h_t$ & $h_t = o_t\tanh(c_t)$ \\
\hline
0 & 0 & 0 & $0=\sigma(-10)\tanh(\sigma(0)\tanh(0))$ \\ 
0 & 1 & 1 & $1=\sigma(20-10)\tanh(\sigma(10+0)\tanh(10+0))$ \\ 
1 & 0 & 1 & $1=\sigma(20-10)\tanh(\sigma(10+0)\tanh(10+0))$\\ 
1 & 1 & 0 & { $0=\sigma(20+20-10)\tanh(\sigma(10+10+0)\tanh(10 +10 +0))$} \\ 
\hline
\end{tabular}\end{center}

\end{comment}



\end{document}
