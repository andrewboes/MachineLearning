% CVPR 2022 Paper Template
% based on the CVPR template provided by Ming-Ming Cheng (https://github.com/MCG-NKU/CVPR_Template)
% modified and extended by Stefan Roth (stefan.roth@NOSPAMtu-darmstadt.de)

\documentclass[10pt,twocolumn,letterpaper]{article}

\usepackage[pagenumbers]{cvpr} % To force page numbers, e.g. for an arXiv version

% Include other packages here, before hyperref.
\usepackage{graphicx}
\usepackage{amsmath}
\usepackage{amssymb}
\usepackage{booktabs}
\usepackage{paralist}
\usepackage{comment}

\usepackage[pagebackref,breaklinks,colorlinks]{hyperref}


% Support for easy cross-referencing
\usepackage[capitalize]{cleveref}
\crefname{section}{Sec.}{Secs.}
\Crefname{section}{Section}{Sections}
\Crefname{table}{Table}{Tables}
\crefname{table}{Tab.}{Tabs.}

\newcommand{\xhdr}[1]{\vspace{3pt}\noindent\textbf{#1}}
%%%%%%%%% PAPER ID  - PLEASE UPDATE
\def\cvprPaperID{*****} % *** Enter the CVPR Paper ID here
\def\confName{AI535}
\def\confYear{S2022}


\begin{document}

%%%%%%%%% TITLE - PLEASE UPDATE
\title{ Using Transfer Learning To Count Recreational Fishing Boats }

\author{Andrew Boes \hspace{30pt} Jesse Swartz \\
Oregon Department of Fish and Wildlife\\
{\tt\small \{andrew.j.boes, jesse.l.swartz\}@odfw.oregon.gov}
}
\maketitle

%%%%%%%%% ABSTRACT
\begin{abstract}
   The Oregon Dept. of Fish and Wildlife (ODFW) has setup cameras at several ports and harbors on the Oregon Coast. Currently this video is reviewed by employees, called samplers, to get counts of boats that are going out fishing. This information, combined with other information, is used to get what is called a Catch Effort Estimation which in turn is used to open or close fisheries. Automating this process would expand the count to all days of the week, broaden it to other ports and free up samplers to spend more time getting biological fish data.
\end{abstract}

%%%%%%%%% BODY TEXT
\section{Background}

While more and more Oregonians go outside, the number of fishing licenses and number of fish available to catch has been declining for the past few years [CITATION NEEDED]. ODFW spends a considerable amount of resources analyzing this behavior to maintain the health of the fish population. Due to recent advances in technology such as better 4G cell phone coverage on the Oregon Coast and SpaceX Starlink satellite broadband internet, the Marine Resources Program (MRP) of ODFW is now able to upload videos at three ports. Previously, these ports were either counted in person or VHS tapes were driven to a central location to be reviewed. Starting in 2022 the ports of Newport, Depot Bay, and Charleston have cameras that upload video every two hours to the cloud. This availability allows us to start working on automating the daily vessel count.\\[-5pt]

\section{Introduction}

We propose using transfer learning to identify boats in images taken from port surveillance cameras. We will then use semantic segmentation and multi-object tracking (MOT) over a series of frames to identify individual boats which will allow us to get the count for a given time period.  As the port sampler count is only recreational vessels, we will use zero-shot transfer learning to remove non-recreational (commercial, charter, Coast Guard, etc.) boats from the count. Once the count is complete we can compare our results to the data that is obtained by port samplers.

\subsection{Classifying}

After the relatively simple task of parsing the video into images, the next thing to do is to perform instance segmentation on each frame of the video. There are many comprehensive comparisons of different detection methods [CITATION NEEDED] but we chose to compare a few that are available as pre-trained PyTorch models on our data, using our hardware. This will allow us to 

\subsection{Recreational vs. Non-recreational}

Here we have a small set of data that was classified by the pros

\subsection{Multi-Object Tracking}

For the most difficult part of this project, MOT, we tried three different approaches. We started by reviewing several tracking algorithms and decided to start with SORT [CITATION NEEDED]. We thought this was a good place to start as the 'S' stands for simple. While the implementation of this algorithm was fairly straight forward, we found that as it relied on Kalman filter it was prone to drifts that were difficult to recover from. We started our second approach.

\section{Related Work}

As stated above, the most complicated part of this project is tracking the object from frame to frame. Recently, tracking-by-detection has become the standard paradigm for MOT. The main idea is to split the problem into two parts: object detection and data association. Over the past few years, object detection has seen great
improvement thanks to deep learning techniques, but data association remains a challenge for multi-object tracking [CITATION NEEDED]. 





\begin{comment}

\begin{figure*}[t]
  \centering
  \begin{subfigure}{0.98\linewidth}
    
    \includegraphics[scale=1.25]{roughFlow.png}
    \caption{An example of a subfigure.}
    \label{fig:short-a}
  \end{subfigure}
  \hfill
\end{figure*}



%------------------------------------------------------------------------




\begin{table}[t]
  \centering
  \begin{tabular}{@{}lc@{}}
    \toprule
    Method & Frobnability \\
    \midrule
    Theirs & Frumpy \\
    Yours & Frobbly \\
    Ours & Makes one's heart Frob\\
    \bottomrule
  \end{tabular}
  \caption{Results.   Ours is better.}
  \label{tab:example}
\end{table}
\end{comment}
%-------------------------------------------------------------------------





%%%%%%%%% REFERENCES
{\small
\bibliographystyle{ieee_fullname}
\bibliography{egbib}
}

\end{document}
